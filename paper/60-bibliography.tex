\section*{СПИСОК ИСПОЛЬЗОВАННЫХ ИСТОЧНИКОВ}
\addcontentsline{toc}{section}{СПИСОК ИСПОЛЬЗОВАННЫХ ИСТОЧНИКОВ}

\begingroup
\renewcommand{\section}[2]{}
\begin{thebibliography}{}
	\bibitem{subj:demand} Милов Сергей Николаевич, Милов Алексей Сергеевич. Исследование проблем управления ассортиментом и товарными запасами в торговых сетях // Вестник РЭА им. Г. В. Плеханова. 2019. №5 (107).
	
	\bibitem{subj:scm} Хашман Т.Т. Управление цепочками поставок. Гуманитарный вестник,
	2013, вып. 10. [Электронный ресурс] Режим доступа: http://www.hmbul.ru/articles/114/114.pdf (дата обращения 24.11.2021)
	
	\bibitem{subj:auto_eff} Костышева Яна Вячеславовна Эффективность применения программных обеспечений в области транспортной логистики // Экономикс. 2013. №1.
	
	\bibitem{subj:main} Макаров М. А., Мартынюк А. В., Зарецкий А. В. Транспортная логистика // Актуальные проблемы гуманитарных и естественных наук. 2012. №12.
	
	\bibitem{subj:small_business} Логистика для малого бизнеса при небольших объёмах [Электронный ресурс]. Режим доступа: https://itctraining.ru/biblioteka/logistika-ved/postroenie-logistiki-pri-nebolshikh-obemakh, свободный (дата обращения 01.12.2021)
	
	\bibitem{subj:tms_cmp} Сборник научных статей по итогам работы Международного
	научного форума НАУКА И ИННОВАЦИИ- СОВРЕМЕННЫЕ
	КОНЦЕПЦИИ (г. Москва, 25 января 2019 г.). / отв. ред. Д.Р.
	Хисматуллин. – Москва: Издательство Инфинити, 2019. – 140 с.
	
	\bibitem{trans:main} А. В. Кузнецов, Н. И. Холод, Л. С. Костевич. Руководство к решению задач по математическому программированию. — Минск: Высшая школа, 1978. — С. 110.
	
	\bibitem{trans:polycrit} С. И. Носков, А. И. Рязанцев. Двухкритериальная транспортная задача // T-Comm: Телекоммуникации и транспорт. 2019. Том 13. С. 59-63
	
	\bibitem{trans:potential} И.В. Романовский. Алгоритмы решения экстремальных задач. М.: Наука, 1977. 352 c.
	
	\bibitem{trans:polyprod} Сеславин А.И., Сеславина Е.А. Оптимизация и математические методы принятия решений. Учебное пособие. – М.: МИИТ, 2011. – 152. 
	
	\bibitem{potential:polyindex} Косенко О.В. дис. РАЗРАБОТКА МЕТОДОВ И АЛГОРИТМОВ РЕШЕНИЯ МНОГОИНДЕКСНЫХ РАСПРЕДЕЛИТЕЛЬНЫХ ЗАДАЧ В УСЛОВИЯХ НЕОПРЕДЕЛЕННОСТИ
канд. техн. наук: 05.13.01. -- «ЮЖНЫЙ ФЕДЕРАЛЬНЫЙ УНИВЕРСИТЕТ» ИНСТИТУТ РАДИОТЕХНИЧЕСКИХ СИСТЕМ И УПРАВЛЕНИЯ (ИРТСУ), Таганрог -- 2017 г. - 172 с.
\end{thebibliography}
\endgroup

\pagebreak