\section{Аналитическая часть}

\subsection{Постановка задачи}
	Необходимо разработать программу, которая по предоставленной информации об элементах транспортной системы составляла бы рекомендации по объёмам и маршрутам поставки.
	Должна быть реализована возможность обновлять текущие данные об объектах системы, а также добавлять новые.

\subsection{Актуальность проблемы}
	В данный момент торговые розничные сети (другое название - ретейл) динамически развиваются и с каждым годом занимают всё большую долю в общем объёме розничной торговли\cite{subj:demand}. Деятельность подобных предприятий тесно связана с различными задачами логистики: планирование и контроль затрат, управление запасами различных ресурсов. Но особое место среди них занимает транспортная логистика. Затраты на неё являются существенными, что обосновывается её сложностью и жизненной важностью для деятельности фирмы. 
	
	Таким образом, расширение сегмента розничных сетей на рынке влечёт за собой повышение спроса на перевозку товаров. Большинство транспортных компаний прибегают к использованию программного обеспечения для ускорения и упрощения различных этапов процесса перевозки\cite{subj:auto_eff}.  Автоматизация данной работы позволяет повысить её эффективность и надёжность.

\subsection{Анализ предметной области}
	Целью деятельности транспортной логистики является организация перемещения груза между двумя местами по оптимальному маршруту\cite{subj:main}. В данном случае оптимальным считается тот маршрут, который позволяет перевезти объекты в предусмотренные сроки (желательно, минимальные) с наименьшими затратами и вредом для них. Стоит отметить, что зачастую возможные маршруты не являются оптимальными сразу по всем критериям, поэтому приходится принимать компромиссные решения.
	
	Случаи, когда задачи розничной торговли, транспортировки и складирования товара (а иногда даже производства) выполняются одной фирмой достаточно редки и характерны только для крупного бизнеса. Этот подход организации позволяет улучшить интеграцию всех перечисленных элементов логистической системы, что способно снизить издержки на каждом из этапов. Малые предприятия не располагают подобными ресурсами и, как правило, пользуются услугами других компаний для данной функции. В рамках данной работы рассматривается решение задач планирования для автомобильной транспортной компании.
	
	Логистика малого бизнеса имеет ряд особенностей\cite{subj:small_business}. Как было отмечено, небольшие организации не способны содержать необходимый штат сотрудников, транспортных средств и т.д. а также не обладают достаточно большим оборотом товаров для собственной организации перевозки. Малые объёмы поставок для каждого отдельного предприятия приводят к тому, что компании-перевозчики совмещают планируют один маршрут сразу через несколько точек доставки.
	
	Можно заключить, что обыкновенный процесс работы фирмы доставки заключается в следующем. Несколько ретейл предприятий формируют заказы с указанием заказываемых товаров и их объёмов. Транспортная компания определяет совместные маршруты и средства доставки, формирует заказ для складского предприятия/ий. Назначенные грузовые автомобили загружаются на складе и развозят груз по нужным пунктам.

\subsection{Описание модели}
	Для дальнейшего анализа требуется сформировать описание модели исследуемой системы. В качестве объектов можно выделить следующее.
	
	\textbf{Товар}. Так как речь идёт об оптовых поставках, то закупки осуществляются целыми тарами. Про них известны такие параметры как стоимость и объём.
	
	\textbf{Транспорт}. В рамках модели будем рассматривать это понятие вместе с водителем. Каждый грузовой автомобиль обладает вместимостью и расход топлива за единицу времени (во время движения). Реальный транспорт также имеет текущий запас хода, определяемый объёмом топлива. Однако контролировать её нет необходимости, так как водитель способен пополнять его до необходимого количества между заказами.
	
	\textbf{Заказ}. Формируется либо ретейл фирмой к транспортной фирме, либо от последних к складу. Содержит в себе перечень требуемых товаров и их количество (кратное вместимости тар), а также срок выполнения.
	
	\textbf{Ретейл фирма}. Формирует заказ. Является пунтом доставки.
	
	\textbf{Складская фирма}. Принимает заказ от транспортной компании и подготавливает груз для доставки.
	
	\textbf{Стоянка}. Является начальным и конечным пунктом каждого маршрута.
	
	Расстояния между объектами системами измеряется в минутах.

\subsection{Формализация задачи}
	Сформулированная задача является задачей поиска оптимального решения, а именно транспортной задачей. \cite{trans:main}. Она решает проблему составления плана перевозок из пунктов отправления в пункты потребления, который будет иметь наименьшие затраты на перевозки. 
	
	В простейшем случае модель транспортной системы рассматривается как множество пунктов производства однородного продукта и множество его потребителей. Известны затраты перевозки одной единицы товара для любой пары производителя и потребителя.
	
	Использование такой модели некорректно, так как она не учитывает следующие факторы:
	\begin{itemize}
		\item Склады и транспортные средства ограничены и обладают конечной вместимостью.
		\item Рассматриваемый магазин оперирует сразу множеством товаров. Это порождает сразу ряд дополнительных обстоятельств. Например, тары имеют различные габариты, что влияет на вместимость транспортного средства.
		\item Рассматриваются только маршруты Производитель - Потребитель, тогда как в рассмотренной модели маршрут может проходить сразу через несколько потребителей. Таким образом и перевозимый одним транспортом груз зависит сразу от нескольких потребителей.
	\end{itemize}
		
	Основным способом решения вопроса множества продуктов сводится к условному разбиению поставщиков и потребителей из общих на работающих только с одним продуктом. Таким образом задача сводится к однопродуктовой с общим для всех разбиений одного пункта ограничением на пропускную способность.
	
	В таком случае рассмотрим математическую модель транспортной задачи для одного продукта.
	\paragraph{Формализация данных}
	Формализуем данные метода. В первую очередь обозначим основные величины: $Vol$ - объём одной тары, $Con$ - стоимость топлива (литр / у.е.)
	
	$A_i$ - склады с запасом продукции в $a_i$, ($i = \overline{1, N_a}$).
	
	$B_i$ - потребители с потребностью продукции в $a_i$, ($i = \overline{1, N_b}$).
	
	Данные об $A$ и $B$ можно объединить в понятие пункта маршрута $P$, где $a_i$ - количество продукта (в случае потребителя отрицательно, по модулю равно потребности), где $i = \overline{1, N_a}$ - склады, $i = \overline{N_a+1, N_b+N_a}$ - потребители.
		
	$T_i$ - транспорт с вместимостью в $c_i$ (в кубометрах) и затратой топлива в $f_i$ (в литр / минуту) ($i = \overline{1, N_t}$). Второй, третий и т.д. рейсы транспорта обозначим как $T_{(k-1)N_t + i}$, будет обладать свойствами $T_i$ ($k \ge 1$ - номер рейса).
	
	Тогда $t_{ij} > 0$ - время перемещения между $P_i$ и $P_j$, $v_{ijk} \ge 0$ - количество товара перевезённое $k$-м транспортом между $P_i$ и $P_j$, $i \ne j, i, j = \overline{1, N_b+N_a}$, $k = \overline{1, N_t}$. Вектор $v$, удовлетворяющий ниже идущим условиям и ограничениям считается \textbf{решением}.
	
	\paragraph{Формулирование условия решения}    
	План перевозок можно считать решением задачи в случае, если поставки удовлетворили всех потребителей. В принятом обобщении пунктов это можно записать как
	\begin{equation}
		a_i + \sum_{j=1}^{N_b+N_a} \sum_{k=1}^{N_t} (v_{jik} - v_{ijk}) \ge 0
	\end{equation}
	
	\paragraph{Формулирование ограничений}     
	Ни на одном из этапов перевозки объём продукта не должен превысить максимальную вместимость транспорта.
	\begin{equation}
		v_{ijk} \cdot Vol \le c_k, \forall i, j \in \overline{1, N_b+N_a}, $k \in \overline{1, N_t}$
	\end{equation}

	Обратные перевозки невозможны
	\begin{equation}
		v_{ijk} > 0 => v_{jik} = 0
	\end{equation}

	Транспорт может въехать и выехать из пункта только одним путём
	\begin{equation}
		\left\lbrace 
		\begin{array}{cols}
			\nexists i, k, j_1, j_2: j_1 \ne j_2, v_{ij_1k} > 0, v_{ij_2k} > 0 \\
			\nexists j, k, i_1, i_2: i_1 \ne i_2, v_{i_1jk} > 0, v_{i_2jk} > 0 
		\end{array}
	\end{equation}	
	
	\paragraph{Формулирование критериев}   
	Оригинальный метод решения транспортной задачи опирается на единственный критерий минимизации стоимости перевозок. Однако, возможно модифицирование алгоритма за счёт составной целевой функции\cite{trans:polycrit}.В качестве наиболее значимых в процессе доставки товаров можно выделить следующие критерии.
	
	Максимизация количества выполненных заказов (т.е. тех, для которых заказ был доставлен и вовремя).
	\begin{equation}
		L_1(v) = ??? \to max
	\end{equation}
	
	Минимизация стоимости всех рейсов.
	\begin{equation}
		L_2(v) = Con \cdot \sum_{i=1}^{N_b+N_a} \sum_{j=1}^{N_b+N_a} t_{ij} \cdot \sum_{k=1}^{N_t} v_{ijk} \to min
	\end{equation}
	
	Минимизация среднего количества пунктов в одном маршруте.
	\begin{equation}
		L_3(v) = ??? \to min
	\end{equation}
	
	Максимизация среднего квадратичного интервала между планируемым времени доставки и требуемым.
	\begin{equation}
		L_4(v) = ??? \to max
	\end{equation}	
	
	Максимизация среднего процента заполненности транспорта после загрузки на складе
	\begin{equation}
		L_5(v) = \sum_{i=1}^{N_b+N_a} \sum_{j=1}^{N_b+N_a} max_{k = \overline{1, N_t}}(v_{ijk}) / c_i \to max
	\end{equation}		
	
	Составим результирующий критерий, учитывающий все вышеперечисленные критерии приведённые к однородному виду
	\begin{equation}
		L(v) = -L_1(v) + L_2(v) + L_3(v) - L_4(v) - L_5(v) \to min
	\end{equation}

	\paragraph{Математическая модель}
	Приведём все описанные формализации в математическую модель рассматриваемой задачи поиска оптимального плана поставок.
	
	\begin{equation}
	\begin{array}{cols}		
		L(v) \to min \\
		\left\lbrace 
		\begin{array}{cols}
			a_i + \sum_{j=1}^{N_b+N_a} \sum_{k=1}^{N_t} (v_{jik} - v_{ijk}) \ge 0 \\
			\\
			v_{ijk} \cdot Vol \le c_k, \forall i, j \in \overline{1, N_b+N_a}, $k \in \overline{1, N_t}$ \\
			v_{ijk} > 0 => v_{jik} = 0 \\
			\nexists i, k, j_1, j_2: j_1 \ne j_2, v_{ij_1k} > 0, v_{ij_2k} > 0 \\
			\nexists j, k, i_1, i_2: i_1 \ne i_2, v_{i_1jk} > 0, v_{i_2jk} > 0 
		\end{array}
	\end{array}
	\end{equation}

\subsection{Метод решения}
	Учитывая перечисленные факторы, в качестве основы для решения сформулированной задачи возможно выбрать метод потенциалов в сетевой постановке. Он является модификацией симплекс-метода, применяющегося для многих оптимизационных задач, в том числе и классической транспортной задачи\cite{trans:potential}.
	
	Преимуществом такого метода является то, что он позволяет создавать транзитные маршруты через пункты потребления и добавлять ограничения на пропускную способность, что необходимо в данном случае. Для этого можно условно представить каждого потребителя складом на время отгрузки транспорта в нём. Вместимость такого склада равна неиспользованному месту в грузовике.
	
	...

\subsection*{Вывод}
	Результатом аналитического раздела стала постановка задачи, освещение актуальности проблемы. Была разработана математическая модель исследуемой системы, определён и описан метод решения.
\pagebreak