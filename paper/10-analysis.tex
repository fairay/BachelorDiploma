\section{Аналитическая часть}

\subsection{Постановка задачи}
Необходимо разработать программу, которая по предоставленной информации об элементах транспортной системы составляла бы рекомендации по объёмам и маршрутам поставки.
Должна быть реализована возможность обновлять текущие данные об объектах системы, а также добавлять новые.

\subsection{Актуальность проблемы}
В данный момент торговые розничные сети (другое название - ретейл) динамически развиваются и с каждым годом занимают всё большую долю в общем объёме розничной торговли\cite{subj:demand}. Деятельность подобных предприятий тесно связана с различными задачами логистики: планирование и контроль затрат, управление запасами различных ресурсов. Но особое место среди них занимает транспортная логистика. Затраты на неё являются существенными, что обосновывается её сложностью и жизненной важностью для деятельности фирмы. 

Таким образом, расширение сегмента розничных сетей на рынке влечёт за собой повышение спроса на перевозку товаров. Большинство транспортных компаний прибегают к использованию программного обеспечения для ускорения и упрощения различных этапов процесса перевозки\cite{subj:auto_eff}.  Автоматизация данной работы позволяет повысить её эффективность и надёжность.

\subsection{Анализ предметной области}
Целью деятельности транспортной логистики является организация перемещения груза между двумя местами по оптимальному маршруту\cite{subj:main}. В данном случае оптимальным считается тот маршрут, который позволяет перевезти объекты в предусмотренные сроки (желательно, минимальные) с наименьшими затратами и вредом для них. Стоит отметить, что зачастую возможные маршруты не являются оптимальными сразу по всем критериям, поэтому приходится принимать компромиссные решения.

Случаи, когда задачи розничной торговли, транспортировки и складирования товара (а иногда даже производства) выполняются одной фирмой достаточно редки и характерны только для крупного бизнеса. Этот подход организации позволяет улучшить интеграцию всех перечисленных элементов логистической системы, что способно снизить издержки на каждом из этапов. Малые предприятия не располагают подобными ресурсами и, как правило, пользуются услугами других компаний для данной функции. В рамках данной работы рассматривается решение задач планирования для автомобильной транспортной компании.

Логистика малого бизнеса имеет ряд особенностей\cite{subj:small_business}. Как было отмечено, небольшие организации не способны содержать необходимый штат сотрудников, транспортных средств и т.д. а также не обладают достаточно большим оборотом товаров для собственной организации перевозки. Малые объёмы поставок для каждого отдельного предприятия приводят к тому, что компании-перевозчики совмещают планируют один маршрут сразу через несколько точек доставки.

Можно заключить, что обыкновенный процесс работы фирмы доставки заключается в следующем. Несколько ретейл предприятий формируют заказы с указанием заказываемых товаров и их объёмов. Транспортная компания определяет совместные маршруты и средства доставки, формирует заказ для складского предприятия/ий. Назначенные грузовые автомобили загружаются на складе и развозят груз по нужным пунктам.

\subsection{Метод решения}
Сформулированная задача является задачей поиска оптимального решения, а именно транспортной задачей. \cite{trans:main}. Она решает проблему составления плана перевозок из пунктов отправления в пункты потребления, который будет иметь наименьшие затраты на перевозки. 

В простейшем случае модель транспортной системы рассматривается как множество пунктов производства однородного продукта и множество его потребителей. Известны затраты перевозки одной единицы товара для любой пары производителя и потребителя.

Использование такой модели невозможно, так как она не учитывает следующие факторы:
\begin{itemize}
	\item Склады и транспортные средства ограничены и обладают конечной вместимостью.
	\item Рассматриваемый магазин оперирует сразу множеством товаров. Это порождает сразу ряд дополнительных обстоятельств:
	\begin{itemize}
		\item продукты имеют различные габариты, что влияет на вместимость транспортного средства;
		\item также у них различается срок годности, спрос и множество других параметров, которые можно объединить понятием приоритета;
	\end{itemize}
	\item Для решения данной задачи достаточно доставить на склад магазина объём товара равный величине спроса. Однако, обязательно следует учитывать страховые запасы, создаваемые на случай резкого увеличения спроса или отмены поставок.
\end{itemize}


В её рамках необходимо создать математическую модель, описывающую транспортную систему. Это включает в себя следующие задачи:
\begin{enumerate}
	\item формирование критериев качества управления
	\item формулировка цели управления выражением через критерии
	\item определение уравнений, описывающих движение объекта управления
	\item определение ограничений на используемые ресурсы
\end{enumerate}

\textbf{Система с сосредоточенными параметрами}



\subsection{Критерии оптимизации}
В качестве наиболее значимых в процессе доставки товаров можно выделить следующие факторы:
\begin{itemize}
	\item стоимость перевозки и хранения
	\item срок доставки до конечного пункта (относительно срока годности)
	\item дефицит товаров на каждом узле
\end{itemize}

\pagebreak