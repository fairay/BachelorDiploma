\section{Технологическая часть}

В данном разделе будет представленно обоснование выбора языка и среды программирования, ...

\subsection{Выбор средств программной реализации}
В качестве языка программирования был выбран Python 3, ввиду следующих факторов.

\begin{itemize}
	\item За время обучения был накоплен существенный опыт в использовании данного средства, что позволит сократить время разработки программы.
	\item Язык поддерживает объектно-ориентированный подход, что полезно при структурировании большого количества схожих объектов, которые были выделены ранее.
	\item Наличие библиотек для создания графического интерфейса, визуализации графов и временных диаграмм, которые необходимы для более наглядной демонстрации работы программы.
\end{itemize}
\qquad
В качестве среды разработки был выбран PyCharm по следующим причинам.
\begin{itemize}
	\item Данная IDE предоставляется бесплатно для пользования в учебном заведении\cite{tech:pycharm}.
	\item Имеется значительный опыт в использовании данной среды разработки.
	\item Представлен удобный набор инструментов для написания, тестирования и отладки кода.
\end{itemize}

\subsection{UML-диаграмма}

\subsection{Формат входных файлов}

\subsection{Интерфейс программы}

\subsection*{Вывод}

\pagebreak