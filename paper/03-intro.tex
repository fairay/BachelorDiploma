\section*{ВВЕДЕНИЕ}
\addcontentsline{toc}{section}{ВВЕДЕНИЕ}

В настоящее время активно идёт процесс замены интеллектуального труда человека на специализированные программы. Зачастую авторматизации подвергается лишь отдельные эпизоды работы человека, отличающиеся однотипностью действий, так как они зачастую просты в программной реализации и позволяют повысить производительность работника, предоставив ему больше времени для решения более сложных задач.

Примером подобной подзадачи является планирование доставки в профессии менеджера по логистике. Решение этой проблемы должно учитывать достаточно большое количество факторов. Подобные задачи удаётся решать при помощи программ, так как с использованием математических методов оптимизации они способны принимать наиболее выгодные решения, с точки зрения выделенных критериев, чего не способен сделать человек.

Целью данной работы является разработка метода и программного продукта для планирования доставки товаров. 
Выделены следующие задачи:
\begin{itemize}
	\item провести анализ предметной области, сформулировать критерии оценки оптимальности решений;
	\item формализовать задание, определить необходимый функционал;
	\item выбрать метод оптимизации;
	\item определить набор необходимых данных и способ их хранения;
	\item разработать программу в соответствии с выделенным функционалом. 
	
\end{itemize}

\pagebreak