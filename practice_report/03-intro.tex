\section*{ВВЕДЕНИЕ}
\addcontentsline{toc}{section}{ВВЕДЕНИЕ}

В настоящее время активно идёт процесс замены интеллектуального труда человека на специализированные программы. Зачастую автоматизации подвергаются лишь отдельные эпизоды работы человека, отличающиеся однотипностью действий, так как они относительно просты в программной реализации и позволяют повысить производительность работника, предоставив ему больше времени для решения более сложных задач.

Примером подобной подзадачи является планирование доставки в профессии логиста. Решение этой проблемы должно учитывать достаточно большое количество факторов. Подобные задачи удаётся решать при помощи программ, так как с использованием математических методов оптимизации они способны принимать наиболее выгодные решения, с точки зрения выделенных критериев, что не всегда способен сделать человек.

Целью преддипломной практики является разработка метода для оптимизации планирования грузоперевозок, \, ранее изученного в рамках \, научно-исследовательской работы.

Выделены следующие задачи:
\begin{itemize}
	\item формализовать задание, определить необходимый функционал программного обеспечения;
	\item определить набор необходимых данных и способ их хранения;
	\item разработать и протестировать программу в соответствии с выделенным функционалом;
	\item провести экспериментальную проверку качества функционирования разработанного метода.	
\end{itemize}

\pagebreak