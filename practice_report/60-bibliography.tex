\section*{СПИСОК ИСПОЛЬЗОВАННЫХ ИСТОЧНИКОВ}
\addcontentsline{toc}{section}{СПИСОК ИСПОЛЬЗОВАННЫХ ИСТОЧНИКОВ}

\begingroup
\renewcommand{\section}[2]{}
\begin{thebibliography}{}
	\bibitem{potential} Герасименко Е. М. Метод потенциалов для определения заданного потока минимальной стоимости в нечетком динамическом графе // Известия ЮФУ. Технические науки. 2014. №4 (153).
	
	\bibitem{potential:transit} Кривопалов В. Ю. РЕШЕНИЕ ТРАНСПОРТНОЙ ЗАДАЧИ С ПРОМЕЖУТОЧНЫМИ ПУНКТАМИ И ОГРАНИЧЕНИЕМ ПО ТРАНЗИТУ //Главный редактор СВ Симак. – С. 28.
	
	\bibitem{potential:polyprod} Цехан О. Б. Моделирование и алгоритмизация одной задачи планирования многопродуктовых перевозок с запрещенным транзитом //Веснік ГрДУ імя Я. Купалы.–Серия. – 2011. – Т. 2. – С. 73-89.
	
	\bibitem{schedule:intervals}  Пиневич Е. В., Ганженко Д.В. МЕТОД ИНТЕРВАЛОВ КАК ИНСТРУМЕНТ ПРИ ОПТИМИЗАЦИИ СОСТАВЛЕНИЯ РАСПИСАНИЯ ГРУЗОПЕРЕВОЗОК // ЗАМЕТКИ УЧЕНОГО - 2021. - С. 317 - 321.
	
	\bibitem{alg:Corman}  Алгоритмы. Построение и анализ : пер. с анг. / Кормен Т., Лейзерсон Ч., Ривест Р. [и др.]. - 3-е изд. - М. : Вильямс, 2018. - 1323 с. : ил.
	
	
	\bibitem{alg:Skiena} Алгоритмы. Руководство по разработке --- 2-е изд.: Пер. с англ. --- СПб.: БХВ-Петербург, 2018. --- 720 с.: ил.
	
	
	\bibitem{tech:pycharm} Бесплатные лицензии для обучения программированию [Электронный ресурс]. Режим доступа: https://www.jetbrains.com/ru-ru/community/education, свободный (дата обращения 09.04.2022)
	
	\bibitem{libs:pyqt} Сайт PyQt [Электронный ресурс]. Режим доступа: https://riverbankcomputing.com/software/pyqt/intro, свободный (дата обращения 09.05.2022)
	
	\bibitem{libs:networkx} Сайт документации Networkx [Электронный ресурс]. Режим доступа: https://networkx.org, свободный (дата обращения 10.05.2022)
	
	\bibitem{libs:plotly} Сайт графической библиотеки Plotly [Электронный ресурс]. Режим доступа: https://plotly.com, свободный (дата обращения 10.05.2022)
	
	\bibitem{libs:unittest} Сайт документации фреймворка для тестирования unittest [Электронный ресурс]. Режим доступа: https://docs.python.org/3/library/unittest.html, свободный (дата обращения 20.04.2022)
	
\end{thebibliography}
\endgroup

\pagebreak